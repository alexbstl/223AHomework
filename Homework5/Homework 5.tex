\documentclass[11pt]{article}
\usepackage[margin=1in]{geometry}
\usepackage[english]{babel}
\usepackage{xr}
\usepackage{multicol}
\usepackage{setspace}
%\onehalfspacing
%\doublespacing

%\usepackage{showkeys}

%Graphs
\usepackage{tikz}
\usetikzlibrary{arrows}

%cheatsheet spacing
\usepackage[compact]{titlesec}

%\titlespacing{\section}{0pt}{*0}{*0}
%\titlespacing{\subsection}{0pt}{*0}{*0}
%\titlespacing{\subsubsection}{0pt}{*0}{*0}
%\usepackage[inline]{enumitem}
%\setlist[itemize]{noitemsep,topsep=0pt,parsep=0pt,partopsep=0pt}

\usepackage{mathrsfs,hyperref, algorithm}%, algorithmic}
\usepackage{algpseudocode}
%\usepackage{harvard}
\usepackage[]{amsmath}
\usepackage{amsthm}
\usepackage{fix-cm}
\usepackage[]{amssymb}
\usepackage[]{latexsym}
%\usepackage[latin1]{inputenc}
\usepackage[right]{eurosym}
\usepackage[T1]{fontenc}
\usepackage[]{graphicx}
\usepackage[]{epsfig}
\usepackage{fancyhdr}
\usepackage{bbm}
\usepackage{pstricks}
\usepackage{multirow}
\usepackage[numbers]{natbib}
\usepackage{subcaption}
%\usepackage{subfiles}
\makeatletter
\renewcommand{\theenumi}{\roman{enumi}}
\renewcommand{\labelenumi}{({\theenumi})}
%\renewcommand{\p@enumi}{theenumi-}
%\renewcommand{\@fnsymbol}[1]{\@alph{#1}}
%\renewcommand{\@fnsymbol}[1]{\@roman{#1}}
\newcommand{\v@r}{\operatorname{VaR}}
\newcommand{\avar}{\operatorname{AVaR}}
\newcommand{\bbr}{\mathbb{R}}
\newcommand{\bbc}{\mathbb{C}}
\newcommand{\var}{\mrm{Var}}

\newcommand{\covar}{\mrm{Covar}}
\newcommand{\bbe}{\mathbb{E}}
\newcommand{\bbn}{\mathbb{N}}
\renewcommand{\P}{\mathbb{P}}
\newcommand{\bbp}{\mathbb{P}}
\newcommand{\bbq}{\mathbb{Q}}
\newcommand{\bbg}{\mathbb{G}}
\newcommand{\bbf}{\mathbb{F}}
\newcommand{\bbh}{\mathbb{H}}
\newcommand{\bbj}{\mathbb{J}}
\newcommand{\bbz}{\mathbb{Z}}
\newcommand{\bba}{\mathbb{A}}
\newcommand{\bbx}{\mathbb{X}}
\newcommand{\bby}{\mathbb{Y}}
\newcommand{\bbt}{\mathbb{T}}
\newcommand{\bu}{\mathbf{u}}
\newcommand{\bx}{\mathbf{x}}
\newcommand{\fn}{\footnote}
%\newcommand{\ci}{\citeasnoun}
\newcommand{\ci}{\cite}
\newcommand{\om}{\omega}
\newcommand{\la}{\lambda}
\newcommand{\tla}{\tilde{\lambda}}
\renewcommand{\labelenumi}{(\roman{enumi})}
\newcommand{\ps}{P}
\newcommand{\pss}{\ensuremath{\mathbf{p}}} %small boldface
\newcommand{\pmq}{\ensuremath{\mathbf{Q}}}
\newcommand{\pmqs}{\ensuremath{\mathbf{q}}}
\newcommand{\pas}{P-a.s. }
\newcommand{\pasm}{P\mbox{-a.s. }}
\newcommand{\asm}{\quad\mbox{a.s. }}
\newcommand{\cadlag}{c\`adl\`ag }
\newcommand{\fil}{\mathcal{F}}
\newcommand{\fcal}{\mathcal{F}}
\newcommand{\gcal}{\mathcal{G}}
\newcommand{\dcal}{\mathcal{D}}
\newcommand{\hcal}{\mathcal{H}}
\newcommand{\jcal}{\mathcal{J}}
\newcommand{\pcal}{\mathcal{P}}
\newcommand{\ecal}{\mathcal{E}}
\newcommand{\bcal}{\mathcal{B}}
\newcommand{\ical}{\mathcal{I}}
\newcommand{\rcal}{\mathcal{R}}
\newcommand{\scal}{\mathcal{S}}
\newcommand{\ncal}{\mathcal{N}}
\newcommand{\lcal}{\mathcal{L}}
\newcommand{\tcal}{\mathcal{T}}
\newcommand{\ccal}{\mathcal{C}}
\newcommand{\kcal}{\mathcal{K}}
\newcommand{\acal}{\mathcal{A}}
\newcommand{\mcal}{\mathcal{M}}
\newcommand{\xcal}{\mathcal{X}}
\newcommand{\ycal}{\mathcal{Y}}
\newcommand{\qcal}{\mathcal{Q}}
\newcommand{\ucal}{\mathcal{U}}
\newcommand{\ti}{\times}
\newcommand{\we}{\wedge}
\newcommand\ip[2]{\langle #1, #2 \rangle}
\newcommand{\el}{\ell}
\newcommand\independent{\protect\mathpalette{\protect\independenT}{\perp}}
\def\independenT#1#2{\mathrel{\rlap{$#1#2$}\mkern2mu{#1#2}}}

\newcommand{\indep}{\independent}

\newcommand{\SCF}{{\mbox{\rm SCF}}}

\newcommand{\Z}{{\bf Z}}
\newcommand{\N}{{\bf N}}
\newcommand{\M}{{\cal M}}
\newcommand{\F}{{\cal F}}
\newcommand{\I}{{\cal I}}
\newcommand{\eps}{\varepsilon}
\newcommand{\G}{{\cal G}}
\renewcommand{\L}{{\cal L}}
\renewcommand{\M}{M}
\newcommand{\f}{\frac}
\newcommand{\Norm}{\mcal{N}}

% Griechisch

\newcommand{\ga}{\alpha}
\newcommand{\gb}{\beta}
\newcommand{\gc}{y}
\newcommand{\gd}{\delta}
\newcommand{\gf}{\phi}
\newcommand{\gl}{\lambda}
\newcommand{\gk}{\kappa}
\newcommand{\go}{\omega}
\newcommand{\gt}{\theta}
\newcommand{\gr}{\rho}
\newcommand{\gs}{\sigma}

\newcommand{\Gf}{\Phi}
\newcommand{\Go}{\Omega}
\newcommand{\Gc}{\Gamma}
\newcommand{\Gt}{\theta}
\newcommand{\Gd}{\Delta}
\newcommand{\Gs}{\Sigma}
\newcommand{\Gl}{\Lambda}
\renewcommand{\gc}{\gamma}
\newcommand{\mrm}{\mathrm}

% Differentiation Integration
\newcommand{\p}{\partial}
\newcommand{\diff}{\mrm{d}}
\newcommand{\iy}{\infty}
\newcommand{\lap}{\triangle}
\newcommand{\nab}{\nabla}


\newcommand{\Dt}{{\Delta t}}

% Calculation
\newcounter{modcount}
\newcommand{\modulo}[2]{%
\setcounter{modcount}{#1}\relax
\ifnum\value{modcount}<#2\relax
\else\relax
\addtocounter{modcount}{-#2}\relax
\modulo{\value{modcount}}{#2}\relax
\fi}
\newcommand{\tablepictures}[4][c]{\begin{tabular}[#1]{@{}c@{}}#2\vspace{0.5cm}\\(\alph{#4}) #3\end{tabular}}
\newcounter{gridsearch}
\newcommand{\tabpic}[2]{
    \stepcounter{gridsearch}
    \modulo{\thegridsearch}{2}
%    \ifnum\strcmp{\modulo{#1}{2}}{1}
    \ifnum\value{modcount}=0
        \tablepictures[t]{#1}{#2}{gridsearch}\\[2.0cm]
    \else
        \tablepictures[t]{#1}{#2}{gridsearch}&~&
    \fi
}


\makeatother
\hyphenation{Glei-chung sto-cha-sti-sche Ge-burts-tags-kind ab-ge-ge-be-nen exi-stie-ren re-pre-sen-tation finanz-markt-aufsicht Modell-un-sicher-heit finanz-markt-risi-ken rung-gal- dier gering-sten} \arraycolsep1mm

\newtheorem{lemma}{Lemma}[section]
\newtheorem{proposition}[lemma]{Proposition}
\newtheorem{theorem}[lemma]{Theorem}
\newtheorem{corollary}[lemma]{Corollary}
\newtheorem{definition}[lemma]{Definition}
\newtheorem{example1}[lemma]{Example}
\newtheorem{rem1}[lemma]{Remark}
\newtheorem{assumption}[lemma]{Assumption}
\newtheorem{alg1}[lemma]{Algorithm}
\newtheorem{me1}[lemma]{Mechanism}
\newtheorem*{thm}{Thm}

%makes the following unslanted
\newenvironment{remark}{\begin{rem1}\rm}{\end{rem1}}
\newenvironment{example}{\begin{example1}\rm}{\end{example1}}
\newenvironment{me}{\begin{me1}\rm}{\end{me1}}
\newenvironment{alg}{\begin{alg1}\rm}{\end{alg1}}

\usepackage{color}
%\newcommand{\red}{\color{red}}

%%%%%%%%%%%%%%%%%%
\newcommand{\notiz}[1]{\textcolor{red}{#1}}
\newcommand{\alex}[1]{\textcolor{olive}{#1}}
\newcommand{\new}[1]{\textcolor{blue}{#1}}
\newcommand{\dom}{{\rm dom\,}}
\newcommand{\Int}{{\rm int\,}}
\newcommand{\cl}{{\rm cl\,}}
\newcommand{\T}{\top}
\newcommand{\diag}{\operatorname{diag}}
\DeclareMathOperator{\Min}{Min}
\DeclareMathOperator{\wMin}{wMin}
\DeclareMathOperator*{\Eff}{Eff}
\DeclareMathOperator*{\FIX}{FIX}
\DeclareMathOperator{\App}{App}
\DeclareMathOperator*{\argmin}{arg\,min}
\DeclareMathOperator*{\argmax}{arg\,max}
\DeclareMathOperator*{\essinf}{ess\,inf}
\DeclareMathOperator*{\esssup}{ess\,sup}
\newcommand\norm[1]{\left\lVert#1\right\rVert}
\newcommand\abs[1]{\left|#1\right|}
\newcommand{\ind}[1]{\mathbbm{1}_{\{#1\}}}
\newcommand{\boldgr}[1]{\boldsymbol{#1}}

%%%Alex Probability (and other)
\newcommand{\pderiv}[3][]{% \deriv[<order>]{<func>}{<var>}
  \ensuremath{\frac{\partial^{#1} {#2}}{\partial {#3}^{#1}}}}
\newcommand{\deriv}[3][]{% \deriv[<order>]{<func>}{<var>}
  \ensuremath{\frac{\diff^{#1} {#2}}{\diff {#3}^{#1}}}}

\renewcommand{\to}{\longrightarrow}
\newcommand{\prob}{\mathcal{P}}
\renewcommand{\P}{\mathcal{P}}
\newcommand{\asto}{\xrightarrow{a.s.}}%{\overset{a.s.}{\to}}
\newcommand{\pto}{\xrightarrow{\P}}%{\overset{\P}{\to}}
\newcommand{\Lp}[1]{\xrightarrow{L^#1}}%{\overset{L^#1}\to}
\newcommand{\dto}{\xrightarrow{\dcal}}
\newcommand{\nto}{\xrightarrow{n \to \infty}}%{\overset{n \rightarrow \infty}{\to}}
\newcommand{\dist}{\textrm{~}}
\newcommand{\eqd}{\overset{\mathcal{D}}{=}}
\newcommand{\pconv}{\xrightarrow{\P}}%{\overset{P}{\to}}
\newcommand{\pspace}{$(\Go,\mathcal{F},\P)$}
\newcommand{\fpspace}{$(\Go,\mathcal{F},\mathcal{F}_t,\P)$}
\newcommand{\E}{\mathbb{E}}
\newcommand{\B}[1]{B_{#1}}
\newcommand{\inquote}[1]{``#1''}
\let\oldref\ref
\renewcommand{\ref}[1]{(\oldref{#1})}
%\newcommand{\lcal}{\mathcal{L}}
\newcommand{\bigpar}[1]{\big( #1 \big)}
\newcommand{\gw}{\go}
\newcommand{\gx}{\xi}
\newcommand{\imp}{\Rightarrow}
\newcommand{\nimp}{\nRightarrow}
\newcommand{\gm}{\mu}
\newcommand{\gp}{\psi}
\newcommand{\seq}[1]{\{#1\}}
\newcommand{\pij}[2]{p_{#1}^{#2}}
\newcommand{\geps}{\epsilon}
\DeclareMathOperator{\sgn}{sgn}
\usepackage{ifxetex,ifluatex}
\usepackage{fixltx2e} % provides \textsubscript
\ifnum 0\ifxetex 1\fi\ifluatex 1\fi=0 % if pdftex
  \usepackage[T1]{fontenc}
  \usepackage[utf8]{inputenc}
\else % if luatex or xelatex
  \ifxetex
    \usepackage{mathspec}
  \else
    \usepackage{fontspec}
  \fi
  \defaultfontfeatures{Ligatures=TeX,Scale=MatchLowercase}
\fi
% use upquote if available, for straight quotes in verbatim environments
\IfFileExists{upquote.sty}{\usepackage{upquote}}{}
% use microtype if available
\IfFileExists{microtype.sty}{%
\usepackage{microtype}
\UseMicrotypeSet[protrusion]{basicmath} % disable protrusion for tt fonts
}{}
\usepackage[margin=1in]{geometry}
\usepackage{hyperref}
\hypersetup{unicode=true,
            pdftitle={Homework 2},
            pdfauthor={Alex Bernstein},
            pdfborder={0 0 0},
            breaklinks=true}
\urlstyle{same}  % don't use monospace font for urls
\usepackage{color}
\usepackage{fancyvrb}
\newcommand{\VerbBar}{|}
\newcommand{\VERB}{\Verb[commandchars=\\\{\}]}
\DefineVerbatimEnvironment{Highlighting}{Verbatim}{commandchars=\\\{\}}
% Add ',fontsize=\small' for more characters per line
\usepackage{framed}
\definecolor{shadecolor}{RGB}{248,248,248}
\newenvironment{Shaded}{\begin{snugshade}}{\end{snugshade}}
\newcommand{\KeywordTok}[1]{\textcolor[rgb]{0.13,0.29,0.53}{\textbf{#1}}}
\newcommand{\DataTypeTok}[1]{\textcolor[rgb]{0.13,0.29,0.53}{#1}}
\newcommand{\DecValTok}[1]{\textcolor[rgb]{0.00,0.00,0.81}{#1}}
\newcommand{\BaseNTok}[1]{\textcolor[rgb]{0.00,0.00,0.81}{#1}}
\newcommand{\FloatTok}[1]{\textcolor[rgb]{0.00,0.00,0.81}{#1}}
\newcommand{\ConstantTok}[1]{\textcolor[rgb]{0.00,0.00,0.00}{#1}}
\newcommand{\CharTok}[1]{\textcolor[rgb]{0.31,0.60,0.02}{#1}}
\newcommand{\SpecialCharTok}[1]{\textcolor[rgb]{0.00,0.00,0.00}{#1}}
\newcommand{\StringTok}[1]{\textcolor[rgb]{0.31,0.60,0.02}{#1}}
\newcommand{\VerbatimStringTok}[1]{\textcolor[rgb]{0.31,0.60,0.02}{#1}}
\newcommand{\SpecialStringTok}[1]{\textcolor[rgb]{0.31,0.60,0.02}{#1}}
\newcommand{\ImportTok}[1]{#1}
\newcommand{\CommentTok}[1]{\textcolor[rgb]{0.56,0.35,0.01}{\textit{#1}}}
\newcommand{\DocumentationTok}[1]{\textcolor[rgb]{0.56,0.35,0.01}{\textbf{\textit{#1}}}}
\newcommand{\AnnotationTok}[1]{\textcolor[rgb]{0.56,0.35,0.01}{\textbf{\textit{#1}}}}
\newcommand{\CommentVarTok}[1]{\textcolor[rgb]{0.56,0.35,0.01}{\textbf{\textit{#1}}}}
\newcommand{\OtherTok}[1]{\textcolor[rgb]{0.56,0.35,0.01}{#1}}
\newcommand{\FunctionTok}[1]{\textcolor[rgb]{0.00,0.00,0.00}{#1}}
\newcommand{\VariableTok}[1]{\textcolor[rgb]{0.00,0.00,0.00}{#1}}
\newcommand{\ControlFlowTok}[1]{\textcolor[rgb]{0.13,0.29,0.53}{\textbf{#1}}}
\newcommand{\OperatorTok}[1]{\textcolor[rgb]{0.81,0.36,0.00}{\textbf{#1}}}
\newcommand{\BuiltInTok}[1]{#1}
\newcommand{\ExtensionTok}[1]{#1}
\newcommand{\PreprocessorTok}[1]{\textcolor[rgb]{0.56,0.35,0.01}{\textit{#1}}}
\newcommand{\AttributeTok}[1]{\textcolor[rgb]{0.77,0.63,0.00}{#1}}
\newcommand{\RegionMarkerTok}[1]{#1}
\newcommand{\InformationTok}[1]{\textcolor[rgb]{0.56,0.35,0.01}{\textbf{\textit{#1}}}}
\newcommand{\WarningTok}[1]{\textcolor[rgb]{0.56,0.35,0.01}{\textbf{\textit{#1}}}}
\newcommand{\AlertTok}[1]{\textcolor[rgb]{0.94,0.16,0.16}{#1}}
\newcommand{\ErrorTok}[1]{\textcolor[rgb]{0.64,0.00,0.00}{\textbf{#1}}}
\newcommand{\NormalTok}[1]{#1}
\usepackage{graphicx,grffile}
\makeatletter
\def\maxwidth{\ifdim\Gin@nat@width>\linewidth\linewidth\else\Gin@nat@width\fi}
\def\maxheight{\ifdim\Gin@nat@height>\textheight\textheight\else\Gin@nat@height\fi}
\makeatother
% Scale images if necessary, so that they will not overflow the page
% margins by default, and it is still possible to overwrite the defaults
% using explicit options in \includegraphics[width, height, ...]{}
\setkeys{Gin}{width=\maxwidth,height=\maxheight,keepaspectratio}
\IfFileExists{parskip.sty}{%
\usepackage{parskip}
}{% else
\setlength{\parindent}{0pt}
\setlength{\parskip}{6pt plus 2pt minus 1pt}
}
\setlength{\emergencystretch}{3em}  % prevent overfull lines
\providecommand{\tightlist}{%
  \setlength{\itemsep}{0pt}\setlength{\parskip}{0pt}}
\setcounter{secnumdepth}{0}
% Redefines (sub)paragraphs to behave more like sections
\ifx\paragraph\undefined\else
\let\oldparagraph\paragraph
\renewcommand{\paragraph}[1]{\oldparagraph{#1}\mbox{}}
\fi
\ifx\subparagraph\undefined\else
\let\oldsubparagraph\subparagraph
\renewcommand{\subparagraph}[1]{\oldsubparagraph{#1}\mbox{}}
\fi

%%% Use protect on footnotes to avoid problems with footnotes in titles
\let\rmarkdownfootnote\footnote%
\def\footnote{\protect\rmarkdownfootnote}

%%% Change title format to be more compact
\usepackage{titling}

% Create subtitle command for use in maketitle
\newcommand{\subtitle}[1]{
  \posttitle{
    \begin{center}\large#1\end{center}
    }
}


\date{\today}
\begin{document}
\title{Homework 4 \\ \large PSTAT 223A \vspace{-2ex}}
\author{Alex Bernstein \vspace{-2ex}}
\maketitle
\section*{Problem 7.1}
Find the generator of the following It\^o diffusions.  Note that $f \in \mathcal{C}^2_b$ in all cases (twice continuously differentiable and bounded).
\begin{enumerate}
\item $d X _ { t } = \mu X _ { t } d t + \sigma d B _ { t }$ \label{eq:a}
\begin{proof}
We know $b(X_t,t)=\mu X_t$ and $\gs(X_t,t)=\gs$, so our generator is:
\begin{align*}
Af(x) = \mu x \deriv{f}{x} +\frac{1}{2}\gs^2 \deriv[2]{f}{x}
\end{align*}
\end{proof}
\item $d X _ { t } = r X _ { t } d t + \alpha X _ { t } d B _ { t }$
\begin{proof}
We have $b(X_t,t)=rX_t $ and $\gs(X_t,t)= \ga X_t$ so
\begin{align*}
Af(x) = rx \deriv{f}{x} + \frac{x^2\ga^2}{2} \deriv[2]{f}{x}
\end{align*}
\end{proof}
\item$d Y _ { t } = r d t + \alpha Y _ { t } d B _ { t }$
\begin{proof}
We have $b(t,X_t)=r$ and $\gs(t,X_t)=\ga Y_t$ so
\begin{align*}
Af(x) = r \deriv{f}{x} + \frac{\ga^2 x^2}{2} \deriv[2]{f}{x}
\end{align*}
\end{proof}
\item $d Y _ { t } = \left[ \begin{array} { c } { d t } \\ { d X _ { t } } \end{array} \right]$ where $X _ { t }$ is as in (\ref{eq:a})
\begin{proof}
Note that
\begin{align*}
\begin{bmatrix}
dt \\ dX_t
\end{bmatrix}=
\begin{bmatrix}
1 \\ \mu X_t
\end{bmatrix} dt +
\begin{bmatrix}
0 \\ \gs
\end{bmatrix} dB_t
\end{align*} so letting $x_2=x$ and $x_1=t$:
\begin{align*}
Af(x) &= \deriv{f}{x_1} + \mu x \deriv{f}{x_2} + \frac{1}{2} \gs^2 \deriv[2]{f}{x_2}\\
&= \deriv{f}{t} + \mu x \deriv{f}{x} + \frac{1}{2}\gs^2 \deriv[2]{f}{x}
\end{align*}
\end{proof}
\item $\left[ \begin{array} { c } { d X _ { 1 } } \\ { d X _ { 2 } } \end{array} \right] = \left[ \begin{array} { c } { 1 } \\ { X _ { 2 } } \end{array} \right] d t + \left[ \begin{array} { c } { 0 } \\ { e ^ { X _ { 1 } } } \end{array} \right] d B _ { t }$
\begin{proof}
\begin{align*}
Af(x) = \deriv{f}{X_1}+X_2 \deriv{f}{X_2} + \frac{1}{2}e^{X_1} \deriv[2]{f}{X_2}
\end{align*}
\end{proof}
\item $\left[ \begin{array} { l } { d X _ { 1 } } \\ { d X _ { 2 } } \end{array} \right] = \left[ \begin{array} { l } { 1 } \\ { 0 } \end{array} \right] d t + \left[ \begin{array} { c c } { 1 } & { 0 } \\ { 0 } & { X _ { 1 } } \end{array} \right] \left[ \begin{array} { l } { d B _ { 1 } } \\ { d B _ { 2 } } \end{array} \right]$
\begin{proof}
\begin{align*}
Af(x) = \deriv{f}{X_1} + \frac{1}{2}\deriv[2]{f}{X_1} + \frac{1}{2}X_1^2 \deriv[2]{f}{X_2}
\end{align*}
\end{proof}
\item $X_t=(X_1,X_2,\ldots,X_n)$ where $$
d X _ { k } ( t ) = r _ { k } X _ { k } d t + X _ { k } \cdot \sum _ { j = 1 } ^ { n } \alpha _ { k j } d B _ { j } ; \quad 1 \leq k \leq n
$$
\begin{proof}
\begin{align*}
Af(x) = \sum_{k=1}^n r_k X_k \deriv{f}{X_k} + \frac{1}{2} \sum_{i=1}^n \sum_{j=1}^n X_i X_j \big( \sum_{k=1}^n \ga_{ik} \ga_{jk} \big) \frac{\diff^2 f}{\diff x_i \diff x_j}
\end{align*}
\end{proof}
\end{enumerate}
 \section*{Problem 7.2}
 Find the It\^o diffusion whose generator is the following:
 \begin{enumerate}
 \item $A f ( x ) = f ^ { \prime } ( x ) + f ^ { \prime \prime } ( x ) ; f \in C _ { 0 } ^ { 2 } ( \mathbf { R } )$
 \begin{proof}
 \begin{align*}
 Af(x) &= b(X_t) \deriv{f}{x} + \frac{1}{2}\gs^2 \deriv[2]{f}{x}\\
 \end{align*}
 so $b(X_t)=1$ and $\gs(X_t)^2 = 2$, so 
 \begin{align*}
 dX_t = dt + \sqrt{2} dB_t
 \end{align*}
 \end{proof}
 \item $A f ( t , x ) = \frac { \partial f } { \partial t } + c x \frac { \partial f } { \partial x } + \frac { 1 } { 2 } \alpha ^ { 2 } x ^ { 2 } \frac { \partial ^ { 2 } f } { \partial x ^ { 2 } } ; f \in C _ { 0 } ^ { 2 } \left( \mathbf { R } ^ { 2 } \right)$ where $c,\ga$ are constants.
 \begin{proof} let $b(x_1) = 1$ and $b(x_2) = cx$, and $\gs=\ga x$ where $x_1=t$ and $x_2=x$.  Then:
 \begin{align*}
 \begin{bmatrix}
 \diff X_1 \\ \diff X_2 
 \end{bmatrix} =
 \begin{bmatrix}
 1 \\ c x_2
 \end{bmatrix} dt +
 \begin{bmatrix}
 0 \\ \ga x_2
 \end{bmatrix} dB_t
 \end{align*}
 \end{proof}
 \item $A f \left( x _ { 1 } , x _ { 2 } \right) = 2 x _ { 2 } \frac { \partial f } { \partial x _ { 1 } } + \ln \left( 1 + x _ { 1 } ^ { 2 } + x _ { 2 } ^ { 2 } \right) \frac { \partial f } { \partial x _ { 2 } }+ \frac { 1 } { 2 } \left( 1 + x _ { 1 } ^ { 2 } \right) \frac { \partial ^ { 2 } f } { \partial x _ { 1 } ^ { 2 } } + x _ { 1 } \frac { \partial ^ { 2 } f } { \partial x _ { 1 } \partial x _ { 2 } } + \frac { 1 } { 2 } \cdot \frac { \partial ^ { 2 } f } { \partial x _ { 2 } ^ { 2 } } ; \quad f \in C _ { 0 } ^ { 2 } \left( \mathbf { R } ^ { 2 } \right)$
 \begin{proof}
 Translating the above two-dimensional process into the It\^o diffusion gives us:
 \begin{align*}
 \begin{bmatrix}
 \diff X_1 \\ \diff X_2
 \end{bmatrix}=
 \begin{bmatrix}
 2 X_2 \\ \log(1+X_1^2 +X_2^2)
 \end{bmatrix} dt +
 \begin{bmatrix}
 X_1 & 1 \\ 1 & 0
 \end{bmatrix}
 \begin{bmatrix}
 dB_1 \\ dB_2
 \end{bmatrix}
 \end{align*}
 \end{proof}
 \end{enumerate}
 \section{Problem 7.4}
 Let $B_t^x$ be a 1-dimensional Brownian Motion starting at $x \in \bbr_+$.  Put $\tau = \inf \left\{ t > 0 ; B _ { t } ^ { x } = 0 \right\}$.
 \begin{enumerate}
 \item Prove $\tau < \infty$ a.s. $\P^x$ for all $x>0$.
 \begin{proof} Let $0<x<k$ for some $k$.  Let $\tau_k = \inf\{t>0 ; B_t = 0 \textrm{ or } B_t=k\}$.  $\tau_k$ is an exit time, so $\P^x(\tau_k < \infty) = 1$.   We apply Dynkin's formula to $f(x) = x$, have $Af(x) = 0$ and let $\P^x(X_{\tau_k}=k)=p_k$:
\begin{align*}
\E^x(X_{\tau_k}) &= x\\
X_{\tau_k} p_k + 0 (1-p_k) &=  x\\
p_k = \frac{x}{X_{\tau_k}} = \frac{x}{k}
\end{align*}
So 
\begin{align*}
\P^x( \tau<\infty) &= \lim_{k \to \infty} \Big(1-\P^x(\tau_k = k)\Big) = \lim_{k \to \infty} \Big(1 - p_k\Big)\\
&= \lim_{k \to \infty} \Big( 1-\frac{x}{k}\Big)= 1
\end{align*}
So $\tau<\infty$ a.s.
\end{proof}
\item Prove that $\E^x(\tau_k)= \infty$ for all $x>0$
\begin{proof}
 We use the same exit time formulation as in the previous part and apply Dynkin's formula to $f(x) = x^2$, and $Af(x) = \frac{1}{2} 2 = 1$:
 \begin{align*}
 \E^x (X^2_{\tau_k})&= x^2 + \E^x \Big(\int_0^{\tau_k} Af(X_s) ds \Big)\\
 &=x^2 + \E^x (\tau_k)\\
 0 \P^x(X_{\tau_k}=0)+k^2 \P^x(X_{\tau_k}=k) &= x^2 +  \E^x (\tau_k)\\
(\text{letting $\P^x(X_{\tau_k}=k)=p_k$}) \qquad \qquad \E^x (\tau_k) &= k^2p_k -x^2
 \end{align*}
Combining with $p_k$ derived in the previous part:
\begin{align*}
\E^x(\tau_k) &= k^2 \frac{x}{k}-x^2 = kx-x^2\\
\E^x(\tau) &= \lim_{k \to \infty} \E^x(\tau_k) = \infty
\end{align*}
as expected.
 \end{proof}
 \end{enumerate}
 \section*{Problem 7.9}
 Let $X_t$ be a geometric Brownian Motion, i.e.
 $$
d X _ { t } = r X _ { t } d t + \alpha X _ { t } d B _ { t } , \quad X _ { 0 } = x > 0,
$$ $B_t \in \bbr$; $r, \ga$ are constants.
\begin{enumerate}
\item Find the generator $A$ of $X_t$ and compute $Af(x)$ when $f(x) = x^\gc$; $x>0$, $\gc$ constant.  
\begin{proof}
\begin{align*}
Af(x) &= r x \deriv{f}{x} + \frac{1}{2} \ga^2 x^2 \deriv[2]{f}{x}\\
A(x^\gc)&=r x \gc x^{\gc-1} + \frac{1}{2} ga^2 x^2 \gc (\gc-1) x^{\gc-2}\\
&= x^\gc \Big( r \gc + \frac{1}{2} \ga^2  ( \gc^2 - \gc) \Big)
\end{align*}
\end{proof}
\item If $r< \frac{1}{2} \ga^2$ then $X_t \to 0$ as $t \to \infty$, a.s. $Q^x$, but what is the probability $p$ that $X_t$, when starting  from $x< R$ ever hits $R$?
\begin{proof}
We will apply Dynkin's Formula with $f(x) = x^{\gc_1}$ where $\gc_1=1-\frac{2r}{\ga^2}$.  Note that solving the above SDE defining the Geometric Brownian Motion with $X_0=x$ gives us:
\begin{align*}
X_t &= xe^{\big(r-\frac{\ga^2}{2}\big)t+ \ga B_t}
\end{align*}
Applying our known value of $\gc_1$ to the generator for $x^{\gc_1}$ gives us:
\begin{align*}
A(x^{\gc_1}) &= x^{\gc_1}\Big( r (1-\frac{2r}{\ga^2}) + \frac{1}{2}\ga^2 (1-\frac{2r}{\ga^2})(-\frac{2r}{\ga^2}) \Big)\\
&= x^{\gc_1}\Big( r (1-\frac{2r}{\ga^2}) + \frac{1}{2}\ga^2 (-\frac{2r}{\ga^2}+\frac{4r^2}{\ga^4}) \Big) \\
&= x^{\gc_1} 0 = 0
\end{align*}
Now, define $\tau_R = \inf\{t>0;  X_t = 0 \text{ or }X_t = R\}$.  This is an exit time, and $0<X_0=x<R$, so $\P^x(\tau_R<\infty)=1$.  Putting this together, we get:
\begin{align*}
\E^x(X_{\tau_R}^{\gc_1}) &= x^{\gc_1}\\
0 \P^x(X_{\tau_R}=0) + R^{\gc_1} \P^x(X_{\tau_R}=R) &= x^{\gc_1}\\
p_R =  \P^x(X_{\tau_R}=R) =\bigg( \frac{x}{R}\bigg) ^{\gc_1}
\end{align*}
as expected.
\end{proof}
\item If $r>\frac{1}{2}\ga^2$ then $X_t \xrightarrow{t \to \infty} \infty$.  Let $\tau = \inf \left\{ t > 0 ; X _ { t } \geq R \right\}$.  Use Dynkin's formula with $f(x) = \log x$, $x>0$ to prove that $$
E ^ { x } [ \tau ] = \frac { \ln \frac { R } { x } } { r - \frac { 1 } { 2 } \alpha ^ { 2 } }
$$
\begin{proof}
Let $\tau_\rho= \inf \{ t>0 X_t = R \text{ or }X_t = \rho \}$ where $0<\rho<x=X_0<R$.  We therefore have that $\tau_{\rho}$ is an exit time and $\P^x(\tau_{\gr}<\infty)=1$.  Applying the the diffusion generator to $\log x$ and integrating that from $0$ to $\tau_\gr$ gives us:
\begin{align*}
A \log(X_s) &= r - \frac{1}{2}a^2\\
\int_0^{\tau_\gr} A \log(X_s) ds &= \int_0^{\tau_{\gr}} \Big(r- \frac{1}{2}a^2 \Big) ds = \tau_\gr \big( r - \frac{1}{2}a^2 \big)
\end{align*}
Applying Dynkin's formula gives to $\log X_{\tau_\gr}$ gives us:
\begin{align*}
\E^x( \log(X_{\tau_\rho})) &= \log(x) + \E^x \bigg(\int_0^{\tau_\gr} A \log(X_s) ds \bigg)\\
&= \log(x) + \big( r - \frac{1}{2}a^2 \big) \E^x(\tau_\gr )\\
\P^x(\log(X_{\tau_\gr}) = \gr) \log \gr + \P^x(\log(X_{\tau_\gr}) = R) \log R &= \log(x) + \big( r - \frac{1}{2}a^2 \big) \E^x(\tau_\gr ).
\end{align*}
Let $p_R = \P^x(X_{\tau_\gr}) = R)$ so $ \P^x(\log(X_{\tau_\gr}) = \gr)=1-p_R$.  We now have:
\begin{align*}
(1-p_R) \log \gr  + p_R \log R  &= \log(x) + \big( r - \frac{1}{2}\ga^2 \big) \E^x(\tau_\gr )\\
\E^x( \tau_\gr) &= \frac{p_R(\log R - \log \gr) +\log \gr -\log x}{r-\frac{1}{2}\ga^2}\\
&= \frac{p_R \log R + (1-p_R) \log \gr - \log x}{r -\frac{1}{2}\ga^2}
\end{align*}
From the previous part, we know:
\begin{align*}
\E(X_{\tau_{\gr}}^{\gc_1})= p_R R^{\gc_1} + (1-p_R) \gr^{\gc_1} = x^{\gc_1}
\end{align*}
so \begin{align*}
p_R = \frac{x^{\gc_1}-\gr^{\gc_1}}{R^{\gc_1}-\gr^{\gc_1}}
\end{align*}
and
\begin{align*}
1-p_R = \frac{R^{\gc_1}-x^{\gc_1}}{R^{\gc_1}-\gr^{\gc_1}}
\end{align*}
Taking
\begin{align*}
\lim_{\gr \to 0} (1-p_R) \log \gr &= \lim_{\gr \to 0} \frac{R^{\gc_1}-x^{\gc_1}}{R^{\gc_1}-\gr^{\gc_1}} \log \gr\\
(\text{ L'Hopital's Rule})&= \lim_{\gr \to 0}  \frac{R^{\gc_1}-x^{\gc_1}}{-{\gc_1}\gr^{\gc_1-1}}\gr^{-1} \\
&= \lim_{\gr \to 0} -\frac{R^{\gc_1}-x^{\gc_1}}{\gc_1 \gr^{\gc_1}}\\
(\text{Note that $\gc_1<0$ so $-\gc_1>0$})&= \lim_{\gr \to 0} \frac{-\gr^{-\gc_1}}{\gc_1}=0.
\end{align*}
Similarly, 
\begin{align*}
\lim_{\gr \to 0} p_R &= \lim_{\gr \to 0} \frac{x^{\gc_1}-\gr^{\gc_1}}{R^{\gc_1}-\gr^{\gc_1}}\\
&= \lim_{\gr \to 0} \frac{x^{\gc_1}\gr^{-\gc_1}-1}{R^{\gc_1}\gr^{-\gc_1}-1} = 1,
\end{align*}

\begin{align*}
\E^x( \tau_\gr)&= \frac{p_R \log R + (1-p_R) \log \gr - \log x}{r -\frac{1}{2}\ga^2}\\
& \xrightarrow{\gr \to 0} \frac{\log R - \log x}{r-\frac{1}{2}\ga^2}\\
&= \frac{\log \big(\frac{R}{x})}{r-\frac{1}{2}\ga^2} = \E^x(\tau)
\end{align*}
as expected.
\end{proof}
\end{enumerate}
\section*{Problem 7.18}
\begin{enumerate}
\item Let $$
d X _ { t } = b \left( X _ { t } \right) d t + \sigma \left( X _ { t } \right) d B _ { t } ; \quad X _ { 0 } = x
$$ be a 1-dimensional It\^o diffusion witch characteristic operator $\acal$.  Let $f \in \ccal^2(\bbr)$ be a solution to the differential equation $$
\mathcal { A } f ( x ) = b ( x ) f ^ { \prime } ( x ) + \frac { 1 } { 2 } \sigma ^ { 2 } ( x ) f ^ { \prime \prime } ( x ) = 0 ; \quad x \in \mathbf { R }
$$ let $(a,b)\subset \bbr$ be an open interval such that $x \in (a,b)$ and put $$
\tau = \inf \left\{ t > 0 ; X _ { t } \notin ( a , b ) \right\}.$$  Assume that $\tau<\infty$ a.s. $\qcal^x$ and define $$
p = \P ^ { x } \left[ X _ { \tau } = b \right].
$$
Use Dynkin's formula to prove that 
$$
p = \frac { f ( x ) - f ( a ) } { f ( b ) - f ( a ) }
$$
\begin{proof}
We know by definition, $\acal f(x) =0$, and $\tau$ is an exit time, so $\P^x(\tau<\infty)=1$.  Applying Dynkin's formula, and noting the expectation on the RHS is $0$:
\begin{align*}
\E^x( f(X_\tau)) &= f(x) + 0\\
f(a) \P^x(X_\tau=a)+f(b) \P^x(X_\tau=b) &= f(x)\\
f(a)(1- p) + f(b) p & = f(x)\\
p &= \frac{f(x)-f(a)}{f(b)-f(a)}
\end{align*}
as expected.
\end{proof}
\item Now specialize to the process $$
X _ { t } = x + B _ { t } ; \quad t \geq 0.
$$
Prove that $$
p = \frac { x - a } { b - a }
$$
\begin{proof}
$X_t = x + B_t$.  Note that $dX_t = dB_t$, so this is a Brownian Motion (that starts at $x \neq 0$).  We therefore have $\acal f(x) = \frac{1}{2}\deriv[2]{f}{x}$.  Letting $f(x)=x$ gives us $\acal f(x) = \acal x = 0$.  We therefore have:
\begin{align*}
x&= \E^x (X_\tau) = b p + a(1-p)\\
p &=\frac{x-a}{b-a}
\end{align*}
as expected
\end{proof}
\item Find $p$ if $$
X _ { t } = x + \mu t + \sigma B _ { t } ; \quad t \geq 0
$$ where $\mu, \gs \in \bbr$ are nonzero constants.
\begin{proof}
$X_t = x + \mu t + B_t$ so $dX_t = \mu dt + dB_t$.  We therefore have $\acal f(x) = \mu \deriv{f}{x} + \frac{\gs^2}{2} \deriv[2]{f}{x}$.  Finding a solution to $\acal f(x) = 0$ gives us the straightforward solution to a homogeneous ODE given by:
\begin{align*}
f(x) = e^{\big( \frac{-2 \mu x}{\gs^2}\big)}+c 
\end{align*} (We let $c=0$ w.l.o.g.)
Applying Dynkin's formula gives us:
\begin{align*}
\E^x (f(X_\tau)) &= f(x) + 0 \\
(1-p)e^{\big( \frac{-2 \mu a}{\gs^2}\big)} &+ p e^{\big( \frac{-2 \mu b}{\gs^2}\big)} = e^{\big( \frac{-2 \mu x}{\gs^2}\big)}
\end{align*} solving for $p$ gives us:
\begin{align*}
p & = \frac{e^{\big( \frac{-2 \mu x}{\gs^2}\big)}-e^{\big( \frac{-2 \mu a}{\gs^2}\big)}}{e^{\big( \frac{-2 \mu b}{\gs^2}\big)}-e^{\big( \frac{-2 \mu a}{\gs^2}\big)}}
\end{align*}
\end{proof}
\end{enumerate}
\end{document}
